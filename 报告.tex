\documentclass{article}
\usepackage[UTF8]{ctex}
\usepackage[tc]{titlepic}
\usepackage{titlesec}
\usepackage{cite}
\usepackage{fancyhdr}
\usepackage{booktabs}
\usepackage{graphicx}
\usepackage{geometry}
\usepackage[section]{placeins}
\geometry{a4paper,scale=0.8}
\pagestyle{fancy}

\lhead{大气污染物扩散的粒子追踪\\\today}
\chead{中国科学技术大学\\数学建模课程}

\rhead{Assignment x\\ {\CTEXoptions[today=old]\today}}
\newcommand{\upcite}[1]{\textsuperscript{\cite{#1}}}

\titleformat*{\section}{\bfseries\Large}
\titleformat*{\subsection}{\bfseries\large}

\title{\bfseries 大气污染物扩散的粒子追踪}
\author{
	梅振源 \quad  71 \quad 学号\\
	侯维铭  \quad  64\quad  学号\\
	苏睿涵  \quad 15 \quad PB22000078
}

\begin{document}
	\maketitle
	\begin{abstract}
		随着工业化进程加快,大气污染物(如PM2.5)的扩散规律受到越来越多关注。特别是在复杂地形和特殊气象条件下,污染物的传输特征亟待研究。事实上,这是一个复杂的过程,涉及到风场,湍流,地形,热力学等多种因素的影响。本文旨在构建含拉格朗日粒子模型,对数风廓线模型与高斯烟羽模型在内的多种模型。基于Matlab软件,结合全球地形数据,考虑涡旋风场,匀速风场,山地地形等多种不同的其余因素追踪模拟粒子扩散。根据模拟结果比对,发现拉格朗日模型可以更好的适应动态过程,相较于高斯烟羽模型能更好地模拟复杂的风场与地形变化,进行非均匀扩散,但相应的计算量要求较高,而高斯烟羽模型模拟的是稳态和均匀风场下的污染物呈现高斯分布的烟羽状,牺牲对复杂动态过程的刻画能力换取了相对较小的计算量。结合多种模型,模拟结果显示:在旋转风场作用下,污染物呈螺旋状扩散;在匀速风场作用下,污染物呈烟羽状;复杂地形显著影响粒子沉积分布,沉积主要集中于地势高差较大的区域等。本研究有助于理解污染物在环境中的迁移机制,并为环境治理和灾害预警提供参考。
	\end{abstract}
	\clearpage
	% \setcounter{secnumdepth}{1}
	\setcounter{section}{1}
	\section*{\centerline{一、前言(问题的提出)}}
	\subsection{子标题1}
	你的问题描述xxxx  \\
	\subsection{子标题2}
	xxxxxxx  
	\setcounter{section}{2}
	\section*{\centerline{二、相关工作}}
	引用文献\upcite{946629}
	\setcounter{section}{3}
	\section*{\centerline{三、问题分析}}
	你的问题分析
	\subsection{分析1}
	\subsection{分析2}
	\setcounter{section}{4}
	\section*{\centerline{四、建模的假设}}
	你的模型假设 
	\subsection{假设1}
	\subsection{假设2}
	
	\setcounter{section}{5}
	\section*{\centerline{五、符号说明}}
	\begin{table}[t]
		\caption{\textbf{符号说明}}%标题
		\centering%把表居中
		\begin{tabular}{ccc}%内容全部居中
			\toprule%第一道横线
			符号&说明&单位 \\
			\midrule%第二道横线 
			$\alpha$ &description of $alpha$ & xx \\
			$\beta$ & description of $beta$ & xx \\
			\bottomrule%第三道横线
		\end{tabular}
	\end{table}
	
	\setcounter{section}{6}
	\section*{\centerline{六、数学模型建立}}
	你建立的数学模型和算法的描述 
	
	\setcounter{section}{7}
	\section*{\centerline{七、结果(与对比)}}
	你的算法产生的结果,如果有多种算法产生的结果,需要做对比。
	\subsection{结果展示}
	引用图片
	
	\subsection{结果对比}
	
	\setcounter{section}{8}
	\section*{\centerline{八、结论}}
	实验产生了何种结论  
	
	\setcounter{section}{9}
	\section*{\centerline{九、问题}}
	解决问题的过程中你是否发现的新的问题,或者算法仍然存在改进的地方?  
	
	\bibliographystyle{ieeetr}
	\bibliography{refer}
	
\end{document}